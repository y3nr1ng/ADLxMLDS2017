%%
%% Copyright 2007, 2008, 2009 Elsevier Ltd
%%
%% This file is part of the 'Elsarticle Bundle'.
%% ---------------------------------------------
%%
%% It may be distributed under the conditions of the LaTeX Project Public
%% License, either version 1.2 of this license or (at your option) any
%% later version.  The latest version of this license is in
%%    http://www.latex-project.org/lppl.txt
%% and version 1.2 or later is part of all distributions of LaTeX
%% version 1999/12/01 or later.
%%
%% The list of all files belonging to the 'Elsarticle Bundle' is
%% given in the file `manifest.txt'.
%%

%% Template article for Elsevier's document class `elsarticle'
%% with numbered style bibliographic references
%% SP 2008/03/01
%%
%%
%%
%% $Id: elsarticle-template-num.tex 4 2009-10-24 08:22:58Z rishi $
%%
%%
\documentclass[final,3p]{elsarticle}

%% Use the option review to obtain double line spacing
%% \documentclass[preprint,review,12pt]{elsarticle}

%% Use the options 1p,twocolumn; 3p; 3p,twocolumn; 5p; or 5p,twocolumn
%% for a journal layout:
%% \documentclass[final,1p,times]{elsarticle}
%% \documentclass[final,1p,times,twocolumn]{elsarticle}
%% \documentclass[final,3p,times]{elsarticle}
%% \documentclass[final,3p,times,twocolumn]{elsarticle}
%% \documentclass[final,5p,times]{elsarticle}
%% \documentclass[final,5p,times,twocolumn]{elsarticle}

\usepackage{hyperref}
\usepackage[capitalize]{cleveref}

%% if you use PostScript figures in your article
%% use the graphics package for simple commands
%% \usepackage{graphics}
%% or use the graphicx package for more complicated commands
\usepackage{graphicx}
%% or use the epsfig package if you prefer to use the old commands
%% \usepackage{epsfig}

\usepackage{caption,subcaption}
\usepackage{float}

%% The amssymb package provides various useful mathematical symbols
\usepackage{amssymb}
%% The amsthm package provides extended theorem environments
%% \usepackage{amsthm}

%% The lineno packages adds line numbers. Start line numbering with
%% \begin{linenumbers}, end it with \end{linenumbers}. Or switch it on
%% for the whole article with \linenumbers after \end{frontmatter}.
%% \usepackage{lineno}
\usepackage{amsmath}
\usepackage{xfrac}

% Modify citation style.
\usepackage[numbers]{natbib}

% Packages for custom table views.
% The multirow package provides merged row cells, while booktabs allows customizing the lines.
\usepackage{multirow, booktabs}
% These packages allow colors in table.
\usepackage{color, colortbl}

% Chinese support
\usepackage{xeCJK}
\setCJKmainfont{Lantinghei TC}

%% natbib.sty is loaded by default. However, natbib options can be
%% provided with \biboptions{...} command. Following options are
%% valid:

%%   round  -  round parentheses are used (default)
%%   square -  square brackets are used   [option]
%%   curly  -  curly braces are used      {option}
%%   angle  -  angle brackets are used    <option>
%%   semicolon  -  multiple citations separated by semi-colon
%%   colon  - same as semicolon, an earlier confusion
%%   comma  -  separated by comma
%%   numbers-  selects numerical citations
%%   super  -  numerical citations as superscripts
%%   sort   -  sorts multiple citations according to order in ref. list
%%   sort&compress   -  like sort, but also compresses numerical citations
%%   compress - compresses without sorting
%%
%% \biboptions{comma,round}

% \biboptions{}

\journal{LS1012 General Biology Lab, 106-1}

% adjust the footer
\makeatletter
\def\ps@pprintTitle{%
 \let\@oddhead\@empty%
 \let\@evenhead\@oddhead
 \def\@oddfoot{\centerline{\thepage}}%
 \let\@evenfoot\@oddfoot}
\makeatother

% custom color
\definecolor{Gray}{gray}{0.9}

\begin{document}

\begin{frontmatter}

\title{作業一 Sequence Labeling}

\author{劉彥廷~B03902036}

\end{frontmatter}

%%
%% Start line numbering here if you want
%%
% \linenumbers

\section{模型敘述}	
	
	\subsection{RNN}
		\begin{figure}[H]
			\centering
			\includegraphics[width=0.5\textwidth]{images/rnn_mfcc}
			\caption{繳交的 RNN 模型} \label{fig:rnn}
		\end{figure}
		Input Layer 使用了以 0 作為遮罩的 Masking Layer 讓模型忽略所有使用 0 在特徵維度進行 padding 的 frames,亦即,如果一個句子長度不足 777 個 frames,會提供各個維度特徵皆為 0 的 frames 來補足。
		
		整體設計上依照作業規格的投影片所指示的 LSTM,總計兩層 Bidirectional 的 LSTM。
		兩層各自有一組的 forward 與 backward 層,各組的單元數分別為 128 $\times$ 777 與 64 $\times$ 777。
		其中 777 這個數字乃為了滿足訓練的資料組最長的句子有 777 個 frames。
		
		Output Layer 由一層 Dense Layer 構成,裡頭的單元數恰好吻合 phones 的數量(48),activation 的方式為 softmax。
		
	\subsection{RNN + CNN}
		TBA
		
\section{優化方式}
	\subsection{策略}
		本次作業採取的策略為
		\begin{enumerate}
			\item 使用 MFCC 與預設的 batch size(32)以及論文 \cite{Graves_2005} 所採用的 10 個 epochs,但選用不同的模型。
			\item 在參考過 RNN 與 LSTM(在 Keras 當中的分類)的 framw-wise 準確率以後,選用 LSTM(爾後增加了 Bidirectional Wrapper)。
			\item 調整 optimizer 與初始化的 kernel(最後沒有依照論文採用的 RandomUniform),並調整 batch size 與 epochs 大小,觀察 loss 變化與 GPU 使用率和訓練時間。
			\item 反覆上述步驟。 
		\end{enumerate}
		
		在這過程當中並沒有使用系統性的在 hyperparameter 的空間(從策略當中,可以調整的主要為 batch size、epochs 與初始化的 kernel)當中搜尋最佳值,僅人工的固定間隔取樣(例:epochs 以 10 為單位從 10 調整到 100 觀察結果)。
	
	\subsection{嘗試過的方法}
		本次作業裡頭嘗試過了
		\begin{itemize}
			\item 1 到 3 層的 GRU
			\item 1 到 3 層的 LSTM
			\item 2 層 LSTM 與 Bidirectional LSTM
			\item Bidirectional LSTM 使用 SGD 與 Adam 兩種不同的 optimizer
			\item Bidirectional LSTM 使用亂數(範圍從 -0.1 到 0.1)與全為 0
		\end{itemize}	
		
		最後決定了維持使用 \cref{fig:rnn} 這個模型基於跨過 baseline 且 50 個 epochs 的訓練時間約莫為 1 小時,允許我嘗試多種 hyperparameter 的設置(batch size、epochs)。
		
\section{結果}
	\subsection{fbank 與 mfcc 比較}
		filter bank 的計算緣起於聲音信號的自然特徵(由不同頻率所組成),合併上人類耳蝸的樣式;MFCC 源起於某些演算法的先天限制,而需要使用 DCT 對 filter bank 的係數做 decorrelate。
		根據 \cite{SpeechPr91:online} 的建議,如果演算法不會受到信號當中彼此高度耦合的現象影響的話,可以選用 filter bank,反之則會建議使用 MFCC。
		從實驗結果當中,在固定模型與 hyperparameters 的情況下,fbank 的結果總會比 MFCC 差了將近 50\% 的 frame-wise 準確率(79\% 與 50\% 的準確率差異)。
		考慮到我們使用了 LSTM 且使用了 Time Distributed Wrapper 讓模型可以針對時序信號自適應,信號本身隨時間的相依性可能會造成模型誤判結果(因為時間上的連續已經在模型當中被考慮了),故 MFCC 是較佳的選擇。
		
	\subsection{epochs 數}
	
	\subsection{模型}
		
%% References
%%
%% Following citation commands can be used in the body text:
%% Usage of \cite is as follows:
%%   \cite{key}         ==>>  [#]
%%   \cite[chap. 2]{key} ==>> [#, chap. 2]
%%

%% References with bibTeX database:
\bibliographystyle{apa}
% \bibliographystyle{elsarticle-num} 
% \bibliographystyle{elsarticle-harv}
% \bibliographystyle{elsarticle-num-names}
% \bibliographystyle{model1a-num-names}
% \bibliographystyle{model1b-num-names}
% \bibliographystyle{model1c-num-names}
% \bibliographystyle{model1-num-names}
% \bibliographystyle{model2-names}
% \bibliographystyle{model3a-num-names}
% \bibliographystyle{model3-num-names}
% \bibliographystyle{model4-names}
% \bibliographystyle{model5-names}
% \bibliographystyle{model6-num-names}

\section{參考文獻}
\bibliography{reference}

%% The Appendices part is started with the command \appendix;
%% appendix sections are then done as normal sections
% Have the appendices start with a new page.
%\newpage
%\appendix
	
\end{document}
